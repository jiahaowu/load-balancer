
%%%%%%%%%%%%%%%%%%%%%%%%%%%%%%%%%%%%%%%%%%%%%%%%%%%%%%%%%%%%%%%%%%%%%%%%%%%%%%%
% @ddblock_begin copyright 
%
% Copyright (c) 1999-2011
% Maryland DSPCAD Research Group, The University of Maryland at College Park 
%
% Permission is hereby granted, without written agreement and without
% license or royalty fees, to use, copy, modify, and distribute this
% software and its documentation for any purpose, provided that the above
% copyright notice and the following two paragraphs appear in all copies
% of this software.
% 
% IN NO EVENT SHALL THE UNIVERSITY OF MARYLAND BE LIABLE TO ANY PARTY
% FOR DIRECT, INDIRECT, SPECIAL, INCIDENTAL, OR CONSEQUENTIAL DAMAGES
% ARISING OUT OF THE USE OF THIS SOFTWARE AND ITS DOCUMENTATION, EVEN IF
% THE UNIVERSITY OF MARYLAND HAS BEEN ADVISED OF THE POSSIBILITY OF
% SUCH DAMAGE.
% 
% THE UNIVERSITY OF MARYLAND SPECIFICALLY DISCLAIMS ANY WARRANTIES,
% INCLUDING, BUT NOT LIMITED TO, THE IMPLIED WARRANTIES OF
% MERCHANTABILITY AND FITNESS FOR A PARTICULAR PURPOSE. THE SOFTWARE
% PROVIDED HEREUNDER IS ON AN "AS IS" BASIS, AND THE UNIVERSITY OF
% MARYLAND HAS NO OBLIGATION TO PROVIDE MAINTENANCE, SUPPORT, UPDATES,
% ENHANCEMENTS, OR MODIFICATIONS.
%
% @ddblock_end copyright 
%%%%%%%%%%%%%%%%%%%%%%%%%%%%%%%%%%%%%%%%%%%%%%%%%%%%%%%%%%%%%%%%%%%%%%%%%%%%%%%

% Latex definitions that accompany \texttt{dicedoc}.

% These definitions provide convenience and standard formatting
% for selected constructs that are relevant to developing project
% documentation. They can be re-defined to provide different
% formatting styles, while adhering to the standard set of
% named formatting constructs that are defined as part of \texttt{dicedoc}.

% \textt{ddcode} 
%
% \texttt{ddcode} is used for inline code segments that do not
% contain special characters like "~". We use \texttt{\verb} for other kinds of
% inline code segments.  We use \texttt{\begin{verbatim} ... \end{verbatim}}
% for single- or multi-line code blocks. One nice thing about \ddcode is that
% macros are enabled.  They are not enabled in the verbatim modes.  The
% \texttt{ddcode} macro takes a single argument, which is the text that is to
% be formatted as inline code.
\newcommand{\ddcode}[1]{\texttt{#1}}


% Standard way to format command usage documentation.
% The macro takes a single argument, which gives the command
% usage (i.e., the list of argument placeholders).
\newcommand{\ddusage}[1]{
\noindent
Usage: \texttt{#1}}

% Some definitions for standard DICE terms.
\newcommand{\dicedoc}{\ddcode{dicedoc}}
\newcommand{\dicelang}{\ddcode{dicelang}}
\newcommand{\dxuser}{\ddcode{dice\_user}}
\newcommand{\dlxuser}{\ddcode{dice\-lang\_\-user}}
\newcommand{\dxstartup}{\ddcode{dice\_startup}}
\newcommand{\dlxstartup}{\ddcode{dicelang\_startup}}
\newcommand{\dxuxdefs}{\ddcode{uxdefs\_dice}}
\newcommand{\dlxuxdefs}{\ddcode{uxdefs\_dicelang}}
\newcommand{\uxarch}{\ddcode{UXARCH}}
\newcommand{\uxdice}{\ddcode{UXDICE}}
\newcommand{\expectedErrors}{\ddcode{ex\-pect\-ed-er\-rors.txt}}
\newcommand{\correctOutput}{\ddcode{cor\-rect-out\-put.txt}}
\newcommand{\dxMtaAgressive}{\ddcode{DX\_MTA\_AGGRESSIVE}}

% Standard formatting for a command (e.g., when writing documentation
% for comman reference manuals or user guides). 
% This assumes that documentation for each command \ddcode{<command>} is stored
% in a separate file (e.g., one that has been generated by dicedoc)
% called \ddcode{<command>.tex}, and that all of the command documentation
% has been placed in or copied into a subdirectory called \ddcode{dicedoc}.
\newcommand{\ddcommand}[1]{
\subsection{\ddcode{#1}} \label{#1}
\input{dicedoc/#1.tex}
}

% Standard formatting to refer readers to relevant examples from the 
% DICE User's Guide.
\newcommand{\ddGuideExamples}[1]{
Relevant DICE User's Guide example(s): \ddcode{#1}. 
}
